% ----------------------------------------
% language, font and text format
% ----------------------------------------
\usepackage[brazilian, american]{babel}
\RequirePackage[utf8]{inputenc}
\usepackage[T1]{fontenc}
\usepackage{amsmath,amssymb,amsfonts}
\usepackage{textcomp}

% ----------------------------------------
% svg
% ----------------------------------------
\usepackage{svg}

% ----------------------------------------
% tables
% ----------------------------------------
\usepackage{makecell}
\usepackage{tabularx}

% ----------------------------------------
% URL with hyperref
% ----------------------------------------
\usepackage{hyperref}

% ----------------------------------------
% includegraphics
% ----------------------------------------
\usepackage{graphicx}
% float larging image with subcaption
\usepackage{float}
\usepackage{newfloat}
\DeclareFloatingEnvironment[fileext=lst,placement={!htbp},name=Listing]{listing}
\usepackage{subcaption}

% ----------------------------------------
% itemize
% ----------------------------------------
\def\labelitemi{--}

% ----------------------------------------
% fixme
% ----------------------------------------
\usepackage[nomargin,inline,final]{fixme}
\fxusetheme{color}

% ----------------------------------------
% tikz-uml
% ----------------------------------------
\usepackage{tikz}
\usepackage{tikz-uml}
\tikzumlset{fill class=white!20, font=\footnotesize\ttfamily}

% ----------------------------------------
% \ref utils
% ----------------------------------------
\newcommand{\fig}[1]{\figurename~\ref{#1}}
\newcommand{\tab}[1]{Table~\ref{#1}}
\newcommand{\lis}[1]{Listing~\ref{#1}}
\newcommand{\appen}[1]{Appendix~\ref{#1}}
\newcommand{\sect}[1]{Section~\ref{#1}}
\newcommand{\subsect}[1]{subsection~\ref{#1}}

% ----------------------------------------
% xml tags utils
% ----------------------------------------
\newcommand{\xml}[1]{\texttt{<#1>}}
\newcommand{\attr}[1]{\emph{#1}}

% ----------------------------------------
% spacing controll
% ----------------------------------------
\newcommand{\captionvspaceend}{\vspace{-1em}}
\newcommand{\captionvspace}{\vspace{-1em}}
\setlength{\itemsep}{0pt}
\setlength{\parskip}{0pt}
\setlength{\parsep}{0pt}
\setlength{\abovecaptionskip}{0.1pt}
\setlength{\belowcaptionskip}{0pt}

% ----------------------------------------
% tree
% ----------------------------------------
\usepackage{tikz}
\usetikzlibrary{trees}

% ----------------------------------------
% minted listing inside tcolorbox
% ----------------------------------------
\usepackage{listings}
\usepackage{tikz}
\usepackage[many]{tcolorbox}
\usepackage{minted}
\tcbuselibrary{minted,breakable,skins,raster}
\newtcblisting[]{mintedlistingtcblisting}[1][!ht]{
    enhanced jigsaw,pad at break*=5mm,
    fontsize=\footnotesize
    colback=yellow!5,colframe=yellow!50!black,listing only,
    listing engine=minted,
    minted language=#1,
    minted options={fontsize=\footnotesize,breaklines,autogobble,linenos,numbersep=3mm},
    overlay={\begin{tcbclipinterior}\fill[red!20!blue!20!white] (frame.south west)
    rectangle ([xshift=5mm]frame.north west);\end{tcbclipinterior}}
}

% ----------------------------------------
% listing code inside tcolorbox
% ----------------------------------------
\usepackage{listings}
\usepackage[many]{tcolorbox}
\tcbuselibrary{listingsutf8}
\newtcblisting[]{mytcblisting}[1][!ht]{
    enhanced jigsaw,
    enforce breakable,
    pad at break*=1mm,
    colback=black!2,
    colframe=black!30,
    top=2mm, bottom=2mm, left=2mm, right=0mm,
    listing only,
    coltitle=black,
    % drop fuzzy shadow,
    listing options={
        % shape
        xleftmargin=0pt, % 0 is default
        xrightmargin=0pt, % 0 is default
        % frame ---
        framesep=0pt,
        aboveskip=-6pt,
        belowskip=-6pt,
        captionpos=b, % sets the caption position
        % numbers ----
        % numberstyle=\footnotesize\ttfamily,
        numbers=left,
        % firstnumber=2
        numbersep=1pt,% how far the line-numbers are from the code
        stepnumber=1, % the step between two line-numbers.
        numberstyle=\tiny\ttfamily,
        % code ----
        language=#1,
        tabsize=2,
        showspaces=false, % show spaces with underscores
        showstringspaces=false, % underline spaces within strings
        showtabs=false, % show tabs using underscores
        breaklines=true, % sets automatic line breaking
        breakindent=8pt,
        basicstyle=\footnotesize\ttfamily,
        columns=fullflexible,
        keywordstyle=\color{black},
        morekeywords={
            <, >,
        },
    },
}